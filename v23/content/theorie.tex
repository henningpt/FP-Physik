\section{Theorie}
\label{sec:Theorie}


\subsection{Analogie von Schallwellen in einem Rohr und einem quantenmechanischen Teilchen in einer Box}
\label{subsec:tch1}

\subsubsection{Stehende Schallwellen in einem Rohr}
\label{subsubsec:tch11}
Breitet sich eine Schallwelle in einem Rohr der Länge $L$ aus, so wird sie an den Rohrenden reflektiert.
Durch Überlagerung hin- und zurücklaufender Wellen, bildet sich eine Stehende Welle aus.
Aufgrund der festgelegten Länge des Rohrs, erfolgt eine konstruktive Interferenz nur für bestimmte Wellenlängen
beziehungsweise Frequenzen. Diese Frequenzen $f$ werden Resonanzfrequenzen genannt und erfüllen die Bedingung


\begin{equation}
  \label{eqn:e1ch1}
  f = \frac{n \cdot c}{2 \cdot L}.
\end{equation}


Dabei bezeichnet $n$ eine natürliche Zahl und $c$ die Schallgeschwindigkeit.\\
Da die Ausbreitung der Schallwellen über eine Veränderung des Luftdrucks erfolgt, lässt sich für den
Druck $p$ folgende Um die Beschreibung der Welle zu vereinfachen, lassen sich einige Randbedingungen
Um die Beschreibung der Welle zu vereinfachen, lassen sich einige Randbedingungen
Differentialgleichung formulieren

\begin{equation}
  \label{eqn:e2ch1}
  \frac{\partial ^2 p}{\partial  t^2} = \frac{1}{\varkappa \cdot \rho} \Delta  p.
\end{equation}
$\varkappa$ bezeichnet die Kompressiblität und $\rho$ die Dichte der Luft.
Um die Beschreibung der Welle zu vereinfachen, lassen sich einige Randbedingungen
festlegen. Aufgrund des vorhandenseins des Rohrs ist direkt klar, dass Geschwindigkeitskomponenten
senkrecht zur Rohrwand verschwinden müssen. Weiterhin lässt sich zeigen, dass der Druck senkrecht zum Rohr
eine räumliche Konstante sein muss.
Mithilfe dieser Randbedingungen lässt sich eine Beschreibung des Drucks,
die kompatibel mit Gleichung \eqref{eqn:e2ch1} ist, durch


\begin{equation}
  \label{eqn:e3ch1}
  p(x) = p_{0} \cdot \cos\left( k\cdot x - \omega \cdot t + \alpha \right)
\end{equation}
finden. Da aber durch Überlagerung von hin- und Rücklaufender Wellen eine stehende Welle ergibt, die es
zu beschreiben gilt, ergibt sich für den Druck letztendlich

\begin{equation}
  \label{eqn:e4ch1}
  p(x) = p_{0} \cdot \cos\left( k \cdot x + \alpha \right) \cdot \cos\left( \omega \cdot t \right).
\end{equation}
Aufgrund der Forderung, dass der Druck an den Rohrenden konstant und damit seine
räumliche Ableitung verschwinden muss, ergeben sich die Bedingungen


\begin{align}
  \alpha &= 0 \\
  k &= \frac{n \cdot \pi}{L}.
\end{align}
Dabei bezeichnet $k$ die Wellenzahl.
Die Dispersionsrelation für Schallwellen lautet


\begin{equation}
  \label{eqn:e7ch1}
  f(k) = \frac{c \cdot k}{2\pi}
\end{equation}


Während eine Welle mit einer Eigenfrequenz des Rohrs ungetrieben fortbestehen kann, ist dies für andere
Frequenzen nicht der Fall. Daher ist dann eine etwas andere Beschreibung notwendig. Für Frequenzen
die einer Eigenfrequenz sehr nah sind, lässt sich sich die Welle gut mit der Bewegungsgleichung
eines getriebenen, gedämpften Oszillators


\begin{equation}
  \label{eqn:e1ch1b}
  \frac{\mathrm{d}^2 p}{\mathrm{d}t^2} + 2 \cdot \gamma \frac{\mathrm{d} p}{\mathrm{d}t} + \omega_{0}^2 \cdot p = K \cdot \cos\left( \omega \cdot t \right)
\end{equation}

charakterisieren.
Hier bezeichnet $\omega_{0}$ die Resonanzfrequenz und $\omega$ die der betrachteten Welle.
Durch Lösen von Gleichung \eqref{eqn:e1ch1b} und der Annahme, dass nur die Amplitude der stationären Lösung
messbar ist, ergibt sich für die frequenzabhängige messbare Amplitude einer Welle , die eine Frequenz sehr nahe
einer Eigenfrequenz besitzt, durch


\begin{equation}
  \label{eqn:e2ch1b}
  |A| = \frac{K}{\sqrt{\left( \omega_{0}^2 - \omega^2 \right)^2 + \left( 2 \gamma \omega \right)}}
\end{equation}
gegeben.
Im tatsächlichen Experiment werden aber nicht nur eine sondern gleich meherer Resonanzen angeregt, sodass
für eine realistischere Beschreibung der Amplitude eine Superposition mehrer solcher Amplituden als Ansatz gewählt werden muss.

\subsubsection{Quantenmechanisches Teilchen in einer Box}
\label{subsubsec:tch12}
Teilchen die der Quantenmechanik folgen, werden im Allgemeinen
durch die Schrödingergleichung


\begin{equation}
  \label{eqn:e8ch1}
  i \hbar \frac{\partial ^2 \psi(r,t)}{\partial  t^2} = -\frac{\hbar^2}{2\cdot m} \Delta  \psi(r,t) + V(r) \psi(r,t)
\end{equation}

beschrieben.
Dabei dient das Betragsquadrat der komplexen Wellenfunktion $\psi(r,t)$
als Wahrscheinlichkeitsdichtefunktion, dafür das Teilchen an
einem bestimmten Ort zu finden.
Bei der Beschreibung eines freien Teilchens, dass sich in einem
1D-Potentialtopf fortbewegt, fällt das Potential $V(r)$
weg und Gleichung \eqref{eqn:e8ch1} wird zu


\begin{equation}
  \label{eqn:e9ch1}
  i \hbar \frac{\partial ^2 \psi(r,t)}{\partial  t^2} = -\frac{\hbar^2}{2\cdot m}
  \frac{\partial ^2}{\partial  x^2} \psi(x, t)
\end{equation}

Gleichung \eqref{eqn:e9ch1} liefert sowohl stationäre\footnote{Stationäre
Wellenfunktionen liefern zeitunabhängige Wahrscheinlichkeitsdichten}
als auch nicht-stationäre Lösungen. Im folgenden werden
nun nur noch die stationären Lösungen behandelt.
Diese werden hier durch


\begin{equation}
  \label{e10ch1}
  E \psi(x,t) = - \frac{\hbar^2}{2 \cdot} \psi
\end{equation}
erzeugt. Mit $E$ wird die Energie des Teilchens bezeichnet.
Über den Ansatz


\begin{equation}
  \label{eqn:e11ch1}
  \psi(x) = A \cdot \sin\left( k \cdot x + \alpha \right),
\end{equation}

sowie der Forderung, dass die Wellenfunktion an den Rändern des
Potentialtopf($x = 0$, $x = L$) verschwindet, ergibt sich
für die Teilchenenergie


\begin{equation}
  \label{eqn:e12ch1}
  E(n) = \frac{\hbar^2 \cdot k(n)^2}{2\cdot m}.
\end{equation}
Dabei bezeichnet $n$ eine natürliche Zahl.\\
In der bisherigen Betrachtung der quantenmechanischen Zustände, wurde nicht berücksichtigt, dass
nur der Grundzustand eines solchen Systems zeitlich stabil ist. Für alle übrigen Zustände gilt, dass
sie mit einer gewissen Rate in den Grundzustand zerfallen. Dieser Zerfall wird nun durch Einbau eines Dämpfungfaktors
in die zeitabhängigen Anteile der Wellenfunktion berücksichtigt. Die Wellenfunktion $\psi$ lässt sich nun als


\begin{equation}
  \label{eqn:e3ch1b}
  \psi(x,t) = f(x) \cdot e^{-\left( \lambda + i \cdot \omega_{0} \right) \cdot t}
\end{equation}
schreiben. Hier bezeichnet $\omega_{0}$ die Frequenz des Grundzustands , $\lambda$ den Dämpfungsfaktor und
$f(x)$ den Raumanteil der Wellenfunktion. Mithilfe einer Fouriertransformation lässt sich die sogenannte
Spektralfunktion $A(\omega)$ gewinnen, deren Betragsquadrat die Wahrscheinlichkeit angibt, das betrachtete Teilchen
mit der zur Frequenz $\omega$ gehörenden Energie zu finden.
Für diese Wahrscheinlichkeit gilt

\begin{equation}
  \label{eqn:e4ch1b}
  |A(\omega)| = \frac{1}{2\pi \cdot \left( \left( \omega_{0}^2 - \omega^2 \right)^2 + \lambda^2 \right)}.
\end{equation}

Aus der Breite der durch \eqref{eqn:e4ch1b} beschriebenen Kurve lässt direkt die Lebensdauer als ihr Kehrwert multipliziert mit $\hbar$ ablesen.
Für die Breite $\Gamma$ gilt

\begin{equation}
  \label{eqn:e5ch1b}
  \Gamma = \hbar \cdot \lambda.
\end{equation}
\subsubsection{Vergleich zwischen Klassik und Quantenmechanik}
\label{subsubsec:tch13}
Im Folgenden werden die Gemeinsamkeiten sowie die Differenzen zwischen
dem im Abschnitt \ref{subsubsec:tch11} klassischen und im Abschnitt
\ref{subsubsec:tch12} quantenmechanischen Fall diskutiert.\\
Zunächst lässt sich feststellen, dass bei beiden Problemen eine
Wellengleichung zur Lösung führt. Dabei ist das Auftreten der
quadratischen Raumableitung identisch, die Zeitableitung hingegen stellt
eine Differenz dar. Letzteres resultiert darin, dass sich im klassischen
Fall eine lineare Dispersionsrelation \eqref{eqn:e7ch1}
und im quantenmechanischen eine quadratische Dispersion \eqref{eqn:e12ch1}
ergibt.\\
Eine weitere Gemeinsamkeit ist, dass in beiden Fällen nur bei bestimmten Frequenzen(Diskretes Spektrum)
die stehenden Wellen ausbilden können. Dabei wird allerdings in der Quantenmechanik ein Energiespektrum
und bei den Schallwellen finden nur für die Frequenzen eine physikalische Betrachtung statt.
Weiterhin besteht ein Unterschied in der physikalischen Interpretation der Wellenfunktion an sich.
Im Fall der klassischen Schallwellen, wird die Oszillation des Luftdrucks und im quantenmechanischen
Fall wird ein Maß für die Aufenthaltswahrscheinlichkeit eines Teilchens untersucht.\\
Um das Analogon zur Lebensdauer eines quantenmechanischen Zustands bei Schallwellen zu finden, lassen sich
Wellen mit Frequenzen nahe Eigenfrequenzen betrachten. In diesem Fall ergeben sich mit \eqref{eqn:e2ch1b} und \eqref{eqn:e4ch1b}
sehr ähnliche Zusammenhänge für die frequenzabhängigen Amplituden.\\ \\
Die Informationen zur Anfertigung dieses Theorieabschnitts wurden der Versuchsanleitung \cite{sample1}
entnommen.


\subsection{Analogie von Schallwellen in einem Kugelresonator und dem quantenmechanischen Wasserstoffatom}
\label{subsec:tch2}

\subsubsection{Das Wasserstoffatom}
\label{subsubsec:tch21}
Das Wasserstoffatom, welches sich aus einem Atomkern und einem Eleketron zusammesetzt, lässt sich aufgrund
seiner Symmetrie mit dem Ansatz

\begin{equation}
  \label{eqn:e1ch2}
  \psi(r,\theta,\phi) = \chi_{l}(r) \cdot Y_{l}^{m}(\theta, \phi)
\end{equation}
für die Wellenfunktion charakterisieren. Dabei bezeichnet $\chi$ den Radialanteil, $Y$ den Winkelanteil, $l$
die Drehimpulsquantenzahl und $m$ die Magenetquantenzahl. Dieser Ansatz löst die Schroedingergleichung für
das Wasserstoffatom

\begin{equation}
  \label{eqn:e2ch2}
  E \psi(\vec r) = - \frac{\hbar^2}{2\cdot m} \Delta  \psi(\vec r) - \frac{e^2}{r} \psi(\vec r)
\end{equation}
wenn man diese in Polarkoordinaten schreibt, dann ergeben sich für $Y$ dei Kugelflächenfunktionen, für die gilt


\begin{equation}
  \label{eqn:e3ch2}
  Y_{l}^{m}(\theta, \phi) \propto P_{l}^{m}(\cos(\theta)) \cdot e^{i\cdot m \cdot \phi}.
\end{equation}
Hier sind mit $P_{l}^{m}$ die Legendre-Polynome gemeint.\\
Erfolgt eine nicht-relativistische Betrachtung, so ergeben sich für die Eigenenergien

\begin{equation}
  \label{eqn:e4ch2}
  E_{n^{'}, l} = -\left( \frac{e^2}{\hbar c} \right)^2 \frac{m \cdot c^2}{2\left(l + 1 + n^{'} \right)}
\end{equation}

Dabei gilt


\begin{align}
  n &= l + n^{'} + 1\\
  0 \leq & l \leq n - 1.
\end{align}

Durch $n$ wird die Hauptquantenzahl beschrieben.


\subsubsection{Stehende Wellen in einem Kugelresonator}
\label{subsubsec:tch22}
Betrachtet man nun einen Kugelresonator, lässt sich eine Beschreibung finden, indem Gleichung
\eqref{eqn:e2ch1} ,durch den Ansatz das die Zeitabhängigkeit durch $cos(\omega t )$ gegeben ist, zur
Helmholtzlgleichung

\begin{equation}
  \label{eqn:e5ch2}
  \omega^2 \cdot p(\vec r) = \Delta  p(\vec r)
\end{equation}
wird.\\
Löst man \eqref{eqn:e5ch2} in Polarkoordinaten ergibt sich als Lösung

\begin{equation}
  \label{eqn:e6ch2}
  p(r, \theta, \phi) = f(r) \cdot Y_{l}^{m}(\theta, \phi).
\end{equation}
Mit dem Radialanteil $f(r)$ und dem Winkelanteil, der sich als die Kugelflächenfunktionen
herausstellt, $Y_{l}^{m}(\theta, \phi)$.\\

Somit ist zu erkennen, dass sowohl im quantenmechanischen Fall des Wasserstoffatoms, als auch bei
den klassischen Schallwellen im Kugelresonator, eine Winkelbeschreibung durch die Kugelflächenfunktionen stattfindet.
Allerdings ergeben sich für den Radialanteil ,aufgrund der andersartigen Ableitungen nach $r$, unterschiedliche
Funktionen. Eine weitere Differenz ist, dass die in Gleichung \eqref{eqn:e4ch2} dargestellten Energien,
Entartung für die gleiche Hauptquantenzahl aufweisen, was im Fall der Schallwellen nicht anwendbar ist.
Allerdings lässt sich für beide Gleichung eine Entartung für die durch $m$, der Magnetquantenzahl($-l \leq m \leq l$)
beschriebene Symmetrie feststellen.\\ \\
Die Informationen zur Anfertigung dieses Theorieabschnitts wurden der Versuchsanleitung \cite{sample2}
entnommen.


\subsection{Simulation eines 1D-Festkörpers}
\label{subsec:tch4}
Anders als im vorherigen Abschnitt \ref{subsec:tch2} dient in der Festkörperphysik der Wellenvektor
$\vec k$ als Quantenzahl. So ergibt sich die Energie $E(k)$ ebenfalls in Abhängigkeit vom Wellenvektor.
In Fall des endlichen Festkörpers, beziehungsweise des endlichen Rohrs ergibt sich eine Diskretisierung
für $k$. Diese folgt

\begin{equation}
  \label{eqn:e1ch4}
  k = \frac{\pi \cdot n}{L}.
\end{equation}
Wobei $n$ eine natürliche Zahl ist. Für den Fall eines näherungsweise unendlich ausgedehnten Körpers,
würde sich ein kontinuierliches Spektrum ergeben.\\
Wie schon in Abschnitt \ref{subsec:tch1} erwähnt, ergibt sich für das freie Elektron eine quadratische
Dispersionsrelation \eqref{eqn:e12ch1}. Wird das Elektron nun in ein periodisches Potential gesetzt,
so verändert sich seine Energiedispersion, wenn seine Wellenlänge ungefähr doppelt so groß wie die
Periodizität des Potentials ist.\\
Im Experiment mit den Schalwellen wird dieses periodische Potential durch Blenden, die periodisch in
das Rohr einebaut werden, realisiert. An diesen wird die Welle dann jeweils wie an einem Potential gestreut.\\ \\
Durch die Einführung des Potentials, ergibt sich im Spektrum eine Bandstruktur, die sich durch Lücken im jeweiligen
Spektrum auszeichnet.
Diese sogenannten Bandlücken treten immer dann auf, wenn die Bragg-Bedingung

\begin{equation}
  \label{eqn:e2ch4}
  \lambda = \frac{2\cdot a}{n}
\end{equation}
zutrifft. Dabei bezeichnet $a$ die Gitterkonstante und $n$ eine natürliche Zahl. Damit lässt sich für die
Gittervektoren des reziproken Gitters $G$ schreiben

\begin{equation}
  \label{eqn:e3ch4}
  G = \frac{2\pi \cdot n}{a}.
\end{equation}
Wobei $n$ eine ganze Zahl darstellt.
Wie auch im direkten Gitter, lassen sich im reziproken Gitter ebenfalls Einheitszellen betrachten. Diese
werden im reziproken Fall als Brillouin-Zonen bezeichnet. In jeder dieser Zonen befinden sich doppelt so viele
Resonanzen wie Einheitszellen. Mithilfe des Bloch-Theorems lässt sich die 1D-Wellenfunktion eines Teilchens
in einem periodischen Potential als

\begin{equation}
  \label{eqn:e4ch4}
  \psi(x) = \sum_{G} C{k-G} \cdot e^{i\left( k - G \right)\cdot x}
\end{equation}
formulieren. Anhand dieser Funktion lässt sich erkennen, dass immer ein Wellenvektor jeder Brillouin Zone
beiträgt. Damit lässt sich begründen die von $k$ abhängige Energiedispersion aussschließlich in der ersten
Brillouin-Zone darzustellen.\\ \\
Die Informationen zur Anfertigung dieses Theorieabschnitts wurden der Versuchsanleitung \cite{sample4}
entnommen.
