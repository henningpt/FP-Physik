\section{Durchführung}
\label{sec:Durchführung}

\subsection{Simulation eines Teilchens im 1D Potentialtopf}
\label{subsec:ch1}
Um ein Teilchen in einem 1D-Potentialtopf zu simulieren,
werden Schallwellen, die in einem Metallrohr stehende Wellen
bilden, erzeugt. Um die Schallwellen zu erzeugen wird
mit der Sounkarte eines Computers ein Signal mit bestimmter
Frequenz, über einen Controller, auf einen Lautsprecher, der am Anfang einer Schiene
befestigt ist, gegeben. Dieses Signal wird gleichzeitig noch auf Kanal A eines Oszilloskops gegeben.
Auf der Schiene wird dann aus verschieden langen Teilstücken
ein Rohr zusammengesetzt an dessen Ende ein Mikrofon
angeschlossen wird. Das Signal des Mikrofons wird mithilfe des Controllers
sowohl an den Computer, als auch an Kanal B des Osziloskops weitergegeben.
Am Computer kann das Signal dann wieder von der Soundkarte interpretiert werden.\\
Zunächst werden für das Rohr $8$ jeweils $\SI{75}{\milli\meter}$
lange Teilstücke verwendet. Für diesen Aufbau wird mithilfe des
Programms \textbf{SpectrumSLC.exe} die Amplitude in Abhängigkeit
von der Frequenz(Spektrum) in einem Bereich von $\SI{100}{\hertz}$ bis
$\SI{10}{\kilo\hertz}$ aufgenommen. Die Schrittweite beträgt
$\SI{10}{\hertz}$ und die Zeit pro Schritt
$\SI{50}{\milli\second}$.
Weiterhin wird für ein Rohr, bestehend aus zwei
$\SI{75}{\milli\meter}$ Teilstücken, ein Spektrum im Bereich
von $\SI{5}{\kilo\hertz}$ bis $\SI{14}{\kilo\hertz}$.
Die Schrittweite beträgt $\SI{5}{\hertz}$ und die Zeit pro Schritt
$\SI{50}{\milli\second}$\footnote{Diese Schrittweite, sowie Zeit pro Schritt werden im
Folgenden immer dann verwendet, wenn für diese Größen
kein anderer Wert angegeben ist.}.



\subsection{Simulation eines Wasserstoffatoms}
\label{subsec:ch2}
Um ein Wasserstoffatom zu simulieren, wird ein
Kugelresonator, in dessen unterer Hälfte ein Lautsprcher
und oberer Hälfte ein Mikrofon eingelassen ist.
Der Resonator ersetzt nun im Aufbau aus Abschnitt
\ref{subsec:ch1} das (zusammengestzte) Metallrohr.
Die obere Hälfte des Resonators lässt sich relativ
zur Unteren drehen. Über eine Winkelskala lässt sich
die relative Rotation zwischen Mikrofon und
Lautsprecher ablesen. Der am Resonator eingestellte Winkel
muss allerdings zur weiteren Rechnung in den Winkel $\theta$,
von dem die physikalische Veränderung ausgeht, über den Zusammenhang

\begin{equation}
  \label{eqn:de1ch2}
  \theta = \arccos\left( \frac{\cos(\alpha) - 1}{2} \right)
\end{equation}
umgerechnet werden. Dies wird allerdings von dem verwendeten Programm übernommen.\\
Zuerst wird ein Winkel von $\SI{180}{\degree}$
eingestellt und ein Spektrum im Bereich von
$\SI{100}{\hertz}$ bis $\SI{10}{\kilo\hertz}$
aufgenommen. Dabei wird in $\SI{10}{\hertz}$ Schritten
bei einer Zeit von $\SI{100}{\milli\second}$ pro Schritt
vorgegangen. Anschließend wird dieser Vorgang mit
Winkeln zwischen $\SI{150}{\degree}$ und $\SI{0}{\degree}$
in jeweils $\SI{30}{\degree}$ Schritten wiederholt.\\
Dann wird für die Winkel $\SI{0}{\degree}$,
$\SI{20}{\degree}$ und $\SI{40}{\degree}$ jeweils
ein Spektrum zwischen $\SI{4,9}{\kilo\hertz}$ und
$\SI{5,0}{\kilo\hertz}$($\SI{0,2}{\hertz}$) Schritte
und Zeit pro Schritt $\SI{100}{\milli\second}$)
aufgenommen.\\
Zuletzt wird noch bei $\SI{180}{\degree}$ ein Spektrum
zwischen $\SI{2}{\kilo\hertz}$ und $\SI{7}{\kilo\hertz}$
aufgenommen. Für jede gemessene Resonanz wird dann
dann die Amplitude für Winkel zwischen $\SI{0}{\degree}$
und $\SI{180}{\degree}$ gemessen und in einem Polarplot
aufgetragen.

\subsection{Simulation eines 1D Festkörpers}
\label{subsec:ch4}

\subsubsection{Simulation eines Teilchens in einem
periodischen 1D-Potential}
\label{subsubsec:ch4a}
Um ein Teilchen in einem periodischen Potential
zu simulieren, werden wieder wie in Abschnitt
\ref{subsec:ch1} stehende Wellen in einem, aus
mehreren Teilstücken zusammengesetzten, Rohr
erzeugt. Allerdings werden die Teilstücke zum Teil durch
Blenden voneinander getrennt. Die Blenden simulieren
dass periodische Potential.
Zunächst wird für verschieden lange Rohre, zusammengesetzt
aus $\SI{75}{\milli\meter}$ langen Teilrohren, jeweils
ein Spektrum zwischen $\SI{6}{\kilo\hertz}$ und
$\SI{9}{\kilo\hertz}$ aufgenommen. Dies wird für eine Anzahl
von Teilrohren von $1$ bis $8$ durchgeführt.\\
Weiterhin wird ein Rohr aus $12$ jeweils
$\SI{50}{\milli\meter}$ langen Teilrohren zusammengesetzt
und dazu ein Spektrum zwischen $\SI{100}{\hertz}$
und $\SI{12}{\kilo\hertz}$ aufgenommen.
Dann wird zwischen jeweils $2$ Teilrohre eine Blende
mit einem Innendurchmesser von $\SI{16}{\milli\meter}$
eingefügt und das Spektrum wird erneut aufgenommen.
Dies wird für ein aus $10$ Teilstücken
zusammengesetztes Rohr wiederholt.
In dem selben Frequenzbereich wird dann noch ein
Spektrum für ein aus$8$ jeweils $\SI{50}{\milli\meter}$
langen Teilrohren zusammengesetztes Rohr aufgenommen.
Dabei wird zwischen zwei Teilrohre jeweils eine
Blende mit Innendurchmesser $\SI{16}{\milli\meter}$
eingebracht.


\subsubsection{Simulation einer 1D Atomkette}
\label{subsubsec:ch4b}
Zur Simulation kann eines Festkörpers kann auch die
Anschauung des 1D Festköpers als eine Kette von Atomen verwendet werden.
Dabei sind die einzelnen Atome mittels einer gewissen Kopplungsstärke
mit einander in Beziehung gesetzt. Diese Kopllung wird mit den im
vorherigen Abschnitt eingeführten Blenden realisiert. Der Versuchsaufbau
gleicht also dem aus Abschnitt \ref{subsubsec:ch4b}\\
Zunächst wird für ein $\SI{50}{\milli\meter}$ langes Rohr
ein Spektrum zwischen $\SI{100}{\hertz}$ bis $\SI{22}{\kilo\hertz}$
aufganommen. Dies wird für ein $\SI{75}{\milli\meter}$ langes Rohr, $2$($3$,$4$,$6$)
$\SI{50}{\milli\meter}$ lange und mit einander über jeweils eine
Blende(jeweils für Innendurchmesser $\SI{10}{\milli\meter}$,
$\SI{13}{\milli\meter}$, $\SI{16}{\milli\meter}$) verbundene Rohre wiederholt.\\
Für ein aus $12$ jeweils $\SI{50}{\milli\meter}$ langen Teilrohren, die
alternierend über ein Blende mit Innendurchmesser $\SI{13}{\milli\meter}$ bzw.
$\SI{16}{\milli\meter}$ gekoppelt sind, ein Spektrum im Bereich von
$\SI{100}{\hertz}$ bis $\SI{12}{\kilo\hertz}$ aufgenommen.\\
Nun wird die Aneinanderkettung mehrer Einheitszellen realiesiert.
Jede Zelle besteht aus der Sequenz: $\SI{50}{\milli\meter}$ langes Teilrohr,
Blende mit Innendurchmesser $\SI{16}{\milli\meter}$, $\SI{75}{\milli\meter}$
langes Teilrohr, Blende mit Innendurchmesser $\SI{13}{\milli\meter}$.
Es werden $5$ Zellen kombiniert und ein Spektrum im Bereich von
$\SI{400}{\hertz}$ bis $\SI{12}{\kilo\hertz}$ vermessen.\\
Zuletzt wird ein sogenannter Defekt in der Struktur des Festkörpers(bzw. der linearen Atomkette)
simuliert. Dazu wird ein Rohr aus $11$ jeweils $\SI{50}{\milli\meter}$
langen Teilrohren und einem Teilrohr einer anderen
Länge\footnote{Dieses Rohr wird im Folgenden als der Defekt bezeichnet.}
zusammengesezt. Getrennt werden zwei Teilrohre immer durch eine
Blende mit Innendurchmesser $\SI{16}{\milli\meter}$.
Der Defekt wird nun an verschiedenen Positionen ($1$, $11$, $7$) eingebaut und
es wird jeweils ein Spektrum im Bereich von $\SI{100}{\hertz}$ bis $\SI{6}{\kilo\hertz}$ aufgenommen.
Diese Messung wird jeweils für einen Defekt der Länge $\SI{25}{\milli\meter}$
und der Länge $\SI{75}{\milli\meter}$ durchgeführt.
