\section{Diskussion}
\label{sec:Diskussion}
Die Ergebnisse zu den stehenden Schallwellen aus Abschnitt \ref{subsec:Stehende Schallwellen} entsprechen größtenteils den klassischen Erwartungen. 
Insbesondere die Reproduzierbarkeit der Messungen von Resonanzfrequenzen und den zugehörigen Amplituden, sowie Peak Breiten, welche durch die Tabellen \ref{tab:1:1} und \ref{tab:1:2} verdeutlicht wird, war zu erwarten.
Auffällig ist, dass die Phasen nicht reproduzierbar sind. 
Hier wurden zumindest gleiche Differenzen zwischen den Phasen der verschiedenen Resonanzfrequenzen erwartet.
Die in Abschnitt \ref{subsec:Der Kugelresonator} angefertigten Übersichtsspektren des Kugelresonators zeigen sehr präzise, dass die Resonanzfrequenzen nicht von dem Winkel, an welchem das Spektrum gemessen wird, abhängen.
Dass sich hingegen die Amplituden an verschiedenen Stellen innerhalb des Resonators unterscheiden liegt daran, dass die Kugelflächenfunktionen im Allgemeinen keine Sphärischen Niveauflächen besitzen, was nach Abschnitt \ref{subsubsec:tch21} bereits bekannt war.
Die Figuren in Abbildung \ref{fig:kugelflaechenfunktionen} entsprechen den aus den Grundvorlesungen bekannten Schnitten durch Kugelflächenfunktionen.
Sie bestätigen daher die Beschreibung der Wellenausbreitung in Kugelresonatoren durch Kugelflächenfunktionen.
Mit einer Abweichung von einem Prozent vom Literaturwert $343.2\text{ms}^{-1}$ \cite{giancoli} wurde die Schallgeschwindigkeit bei Raumtemperatur recht präzise bestimmt, wenn beachtet wird, dass dies nicht die Zielsetzung des Versuchsaufbaus war. 
Eine höhere Genauigkeit wäre zu erreichen, indem die Messintervalle der Frequenz kleiner gewählt werden. 
In Abschnitt \ref{subsec:Modell eines eindimensionalen Festkörpers} wurde außerdem die Linearität der Dispersionsrelation von Schallwellen nachgewiesen.
Dieses in Abbildung \ref{fig:4_2_2} dargestellte Ergebnis zeigt einen eindeutigen Unterschied zwischen stehenden Wellen und ihrem quantenmechanischen Analog.
Eine Ähnlichkeit hingegen zeigt sich in der Aufspaltung der Energien in der Simulation eines Festkörpers durch Rohre, die mit Blenden voneinander getrennt sind.
Dass die Anzahl der Niveaus gerade die Anzahl der Einheitszellen ist konnte größtenteils verifiziert werden. 
Da jedoch die Dichte der Resonanzen, wie in Abbildung \ref{fig:dos} zu sehen ist, am Rand der Bänder verhältnismäßig hoch ist, konnten nicht immer alle Niveaus identifiziert werden.
Wie verschiedene Superstrukturen das Spektrum beeinflussen wird an den Abbildungen \ref{fig:4b_5_12_alternating} und \ref{fig:4b_6} deutlich. 
Die Bänder spalten sich weiter auf; Dabei wird das System schon bei einfachen Strukturen schnell sehr komplex.
Überraschend ist, dass kein Effekt der Defekte auf die Versuchsanordnung mit 12 Rohren der Länge 50~mm feststellbar ist.
Es wird erwartet, dass zusätzliche Resonanzstellen im Bereich der Resonanzen der Defekt-Rohrlänge auftreten. 
Diese werden jedoch nicht beobachtet.
Da die Symmetrie der Anordnung so gebrochen wird, wären auch chaotische Effekte im Rahmen des Möglichen gewesen.
Die Ergebnisse zeigen jedoch, dass der Einfluss der Defekte nahezu vernachlässigbar ist.