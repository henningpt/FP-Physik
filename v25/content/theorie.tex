\section{Theorie}
\label{sec:Theorie}
Der Stern-Gerlach-Versuch dient dazu nachzuweisen,
dass bei Elektronen genau 2 Einstellmöglichkeiten für ihren
Eigendrehimpuls, den Spin $ m_\text{s} $, existieren.
Dabei werden die Werte $ m_\text{s} = \pm \frac{1}{2} $
angenommen.\\ Um diese Gegebenheit zu zeigen, wird im Experiment ein
Strahl von Kaliumatomen durch ein räumlich inhomogenes Magnetfeld
geschickt. Dabei kommt es zu einer Aufspaltung des Kaliumstrahls
in zwei Teile. Diese Aufspaltung kommt dadurch zustande, dass auf die
Atome eine magnetische Kraft $F$ wirkt, die, je nach Orientierung des
Spins des Elektrons auf der äußeren Schale, entweder eine Ablenkung in
Magnetfeldrichtung oder in entegengesetzter Richtung bewirkt.
Eine solche Kraft wirkt immer dann auf ein Atom, wenn es ein magnetisches
Moment $\mu$ besitzt. Dieses setzt sich gemäß
\begin{equation}
  \label{eqn:mag_moment}
  \mu = \mu_{I} + \mu_{L} + \mu_{S}
\end{equation}
zusammen. Dabei bezeichnen die Indizes jeweils die ursächlichen
Drehimpulse. Durch $I$ wird der Kernspin beschrieben, durch $L$ der
Bahndrehimpuls und durch $S$ der Spin. In dem hier beschriebenen
Anwendungsfall kann das aus dem Kernspin resultierende Moment $\mu_{I}$
als vernachlässigbar klein gegenüber den übrigen angenommen werden.
Da innerhalb einer "Elektronenschale" die Drehimpulsquantezahlen $l$
und $s$ symmetrisch besetzt werden, tragen nur die Drehimpulse aus
unvollständig besetzten Schalen effektiv bei. Im Fall des Kaliums
befindet sich die in der einzigen unbesezten Schale, der Äußeren, nur
ein Elektron, dessen Bahndrehimpuls $L$ gleich $0$ ist. Das magnetische
Moment des Kaliumatoms ist also nahezu ausschließlich
durch den Spin des äußeren Elektrons festgelegt. Dadurch reduziert sich
Zusammenhang \eqref{eqn:mag_moment} auf \\ \\
\begin{equation}
  \label{eqn:spin_moment}
  \symbf{\mu_{S}} = \frac{-e}{2 m} \cdot g_{S} \cdot \symbf{S}.
\end{equation}
Dabei bezeichnet $\symbf{S}$ den Spin und $g_{S}$ das zugehörige
gyromagnetische Verhältnis, welches für Elektronen im Allgemeinen
einen Wert von ungefähr $2$ annimmt.\\ \\
Definiert man die Richtung des Magnetfelds entlang der z-Achse eines
3-dimensionalen Koordinatensystem, so ergibt sich die in dieser Richtung
relevanten Komponente des magnetischen Momenents
\begin{equation}
  \label{eqn:z_moment}
\mu_{s_\text{z}} = -g_{S} \cdot m_\text{s} \cdot \mu_\text{B}.
\end{equation}
Dabei bezeichnet $\mu_\text{B}$ die Natur-Konstante die als das Bohr'sche
Magneton bezeichnet wird.
Mithilfe von Gleichung \eqref{eqn:z_moment} lässt sich die Kraft bestimmen,
mit der die Kaliumatome in z-Richtung abgelenkt werden. Es ergibt sich\\ \\
\begin{equation}
  \label{eqn:z_kraft}
  F_\text{z} = - m_{s_{z}} \mu_\text{B} \cdot \partial_{z}B .
\end{equation}
