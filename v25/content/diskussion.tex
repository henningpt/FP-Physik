\section{Diskussion}
\label{sec:Diskussion}
Die in Abschnitt \ref{sec:Auswertung} aufgeführten Abbildungen zeigen alle die erwartete Aufsplittung des Teilchenstrahls.
Auffällig ist, dass jeweils das linke Intensitätsmaximum eine höhere Intensität besitzt als das rechte.
Dies könnte mit dem Zeemann-Effekt zusammen hängen, der für eine Aufspaltung der Energieniveaus des äußeren Elektrons sorgt.
Auf welche Weise die Energie des äußeren Elektrons dafür sorgen könnte, dass eine Ablenkungsrichtung bevorzugt wird, und ob eine solche Wechselwirkung von messbarer Größenordnung ist, müssen weitere Untersuchungen klären.
Der berechnete Wert für das Bohr'sche Magneton $\mu_B = \SI{6.59 \pm 0.25 e-24}{\joule\per\tesla}$ weicht um $28,9\%$ vom Literaturwert $\mu_{B,\text{Lit}} = \SI{9.274e-24}{\joule\per\tesla}$ \cite{nist} ab.
Dieser Wert spiegelt die Größenordnung des Bohr'schen Magnetons sehr gut wieder.
Vor allem im Bereich der Intensitätsmaxima wurden die Intensitätsverteilungen sehr detailliert aufgezeichnet.
Daher wird die Ursache für die Abweichung nicht im Messvorgang gesehen.
Um eine höhere Präzision in der Auswertung zu erzielen müsste eine bessere Methode gefunden werden, um die Intensitätsverteilungen durch eine Fitfunktion anzunähern.
Dazu sind jedoch vertiefte Kenntnisse der nötigen numerischen Methoden erforderlich.