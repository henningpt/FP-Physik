\section{Diskussion}
\label{sec:Diskussion}
Für die mittlere Lebensdauer von Myonen wurden die Werte
$\SI{2.12+-0.09}{\micro\second}$(Fit an Messwerte abzüglich Untergrundrate) und
$\SI{2.13+-0.11}{\micro\second}$(Fit an Messwerte mit Untergrundrate als
Paramter) ermittelt.\\
Diese Werte weichen um $\SI{4+-5}{\percent}$ beziehungsweise
$\SI{3+-5}{\percent}$ vom Literaturwert\cite{lebensdauer} ab.
Es werden somit ähnliche Werte für die mittlere Lebensdauer gefunden. Dies liegt,
daran, dass sich beide Methoden von der Herangehensweise her, nur bei der Bestimmung des Untergrunds
unterscheiden und sich für diesen mit $\SI{3.549+-0.004}{}$ (berechnet) und $\SI{2.71+-3.90}{}$ (Fit-Parameter) ebenfalls ähnliche Werte ergeben.\\ \\
Um generell die Abweichung der Werte vom Literaturwert zu erklären, lässt
sich die endliche Zeitauflösung des Vielkanalanalysators nennen. Dadurch
kommt es zu Fehl- bzw. Mehrfachzuordnungen gemessener Zeiten.\\
Weitere Fehlerquellen können in der Justage der Messapparatur auftreten,
da für Verzögerungszeit nie perfekte Werte eingestellt werden können. So kommt es zu Nicht-
beziehungsweise Fehl-erzeugung von Messereignissen. Dabei spielen vor allem auch nicht ideale elektronische Bauteile eine Rolle.
Zwar wird das in der Auswertung \ref{sec:Auswertung} beschriebene
Problem des Fehlauslösens eines Stopsignals berücksichtigt, dennoch kann
keine der bestimmten Untergrundraten exakt diesen systematischen Fehler
korrigieren. In den Abbildungen \ref{fig:hist} und \ref{fig:hist2} fällt auf, dass einige wenige Messwerte, deren Counts im Bereich von ca. 800 liegen,
vergleichsweise weit von der Regressionsgeraden entfernt sind. Um zu überprüfen ob es sich dabei um für das Experiment sinnvolle Messungen handelt, kann
die Anzahl an Stopsignalen mit der Anzahl an histogrammierten Ereignissen verglichen werden.\\
Es wurden $\SI{25537+-51}{}$ Stopsignale gemessen und im Histogramm wurden $\SI{23080+-150}{}$
Ereignisse gezählt. Somit lässt sich dieser Vergleich nicht als Begründung für die unerwartet hohe Anzahl an Counts bei den von der Regressions-Funktion abweichenden
Messwerten anführen. \\ \\
Abschließend lässt sich die Messung als Bestätigung des Werts der Lebensdauer von Myonen bezeichnen, da in beiden Fällen Werte gefunden wurden, innerhalb
deren Unsicherheiten der Literaturwert liegt. Weiterhin ist zu erwähnen, dass beide Werte nach unten abweichen, was damit zu begründen ist, dass der Literaturwert
für Myonen im Vakuum gilt. Da aber in diesem Experiment Myonen in Materie untersucht werden, wird für die mittlere Lebensdauer ein geringerer Wert erwartet, da
sich positiv geladene  und negative geladene Myonen in Materie anders verhalten. Im Unterschied zum Myon positiver Ladung können negativ geladene Myonen anstatt zu
zerfallen zusammen mit einem Atomkern ein myonisches Atom bilden. Dadurch wird ebenfalls ein Signal im Detektor ausgelöst was aber nicht zur Messung der
eigentlichen mittleren Lebensdauer eines Myons führt.
