\newpage
\section{Auswertung}
\label{sec:Auswertung}

\subsection{Bestimmung der Aktivierungsenergie aus dem anfänglichen Verlauf der Depolarisationskurve}
\label{subsec:method1}
Da sich der Zusammenhang für die Depolarisationskurve zu Beginn durch die Gleichung \eqref{eq:t:11} in guter Näherung
beschreiben lässt, ergibt sich für den logarithmierten Depolarisationsstrom in Abhängigkeit von $T^{-1}$ der Zusammenhang
\\
\begin{equation}
  \label{eqn:lineardepolar}
  \ln{I} = \frac{p^2 \cdot E N_\text{p} }{3 \cdot k_{b}
   \cdot T_\text{p} \cdot \tau_{0}} -\frac{W}{k_{\text{B}}} \cdot T^{-1}
\end{equation}

Aus der Steigung des linearen Zusammenhangs \eqref{eqn:lineardepolar} lässt sich erkennen, dass sich
die Aktivierungsenergie $W$\footnote{In Elektronenvolt} mit Hilfe eines linearen Fits der ersten Messwerte über
\\
\begin{equation}
  \label{eqn:W}
  W = -s \cdot k_\text{B} / e
\end{equation}
bestimmen lässt. Dabei bezeichnet $s$ die Steigung der Fit-Funktion, $k_\text{B}$ die Boltzmann-Konstante und $e$ die Elementarladung.\\ \\
Anhand der in \ref{tab:linfit1} beziehungweise \ref{tab:linfit2} Wertepaare ergeben sich über einen unter \textbf{Python 3}
durchgeführten linearen Fit, der mit der Funktion \textbf{curveFit} aus
dem Paket \textbf{scipy} durchgeführt wurde, die Energien
\\
\begin{align*}
  W_{1} &= \SI{1.66+-0.09}{\electronvolt} \\
  W_{2} &= \SI{0.870+-0.011}{\electronvolt}.
\end{align*}
Dabei wurden die Fit-Parameter
\\
\begin{align*}
  s_{1} &= \SI{-19300+-1000}{\kelvin}\\
  s_{2} &= \SI{-10100+-130}{\kelvin}\\
  b_{1} &= \SI{53+-4}{}\\
  b_{2} &= \SI{16.6+-0.5}{}
\end{align*}
berechnet.
Die Fehler wurden anhand der von \textbf{curveFit} berechneten Kovarianzmatrix bestimmt. Alle Fehlerfortpflanzungen in diesem Versuchsprotokoll
werden mit dem Paket \textbf{uncertainties}\cite{uncertainties} durchgeführt.\\ \\
Die jeweiligen relativen Abweichungen der bestimmten Aktivierungsenergien vom Literaturwert \cite{literaturwert}\\ \\
\begin{align*}
  W_\text{literatur} = \SI{0.66}{\electronvolt}
\end{align*}
betragen
\begin{align*}
  \delta W_{1} &= \frac{\lvert W_{1} - W_\text{literatur} \rvert}{W_\text{literatur}} &= \SI{152+-13}{\percent}\\
  \delta W_{2} &= \frac{\lvert W_{2} - W_\text{literatur} \rvert}{W_\text{literatur}} &= \SI{31.9+-1.7}{\percent} .
\end{align*}
Graphische Darstellungen der linearen Fits\footnote{Fits zur Bestimmung der Aktivierungsenergien.} durch die dafür verwendeten Messwerte
sind in Abbildungen \ref{fig:lin1} und \ref{fig:lin2} gegeben.\\ \\

Die Aufgenommenen Depolarisationskurven sind in Abbildungen \ref{fig:TI1} und \ref{fig:TI2} gezeigt. Da sich zusätzlich zum ersten Maximum noch weitere anschließen und die
Kurve nicht wie im theoretischen Idealfall auf $0$ abfällt, wurde zur Korrektur der gemessenen Werte, wie in \ref{fig:TI1} und \ref{fig:TI2} dargestellt,
jeweils eine lineare Korrektur-unktion von den Meswerten abgezogen.
Dabei wurden weiterhin die Werte die anschließend unterhalb von $0$ lagen für weitere Berechnungen ausgeschlossen.\\
Für die Fit-Paramter der linearen Korrekturfunktionen werden die Werte
\begin{align*}
  s_\text{korrektur 1} &= \SI{1.778+-0.018}{\pico\ampere\per\kelvin}\\
  s_\text{korrektur 2} &= \SI{0.395+-0.009}{\pico\ampere\per\kelvin}\\
  b_\text{korrektur 1} &= \SI{-0.405+-0.005}{\nano\ampere}\\
  b_\text{korrektur 2} &= \SI{-0.763+-0.025}{\nano\ampere}
\end{align*}
ermittelt.\\ \\
Die für den Fit jeweils verwendeten Messwerte sind in Tabelle \ref{tab:korrekturfit} angegeben.
\begin{figure}
  \centering
  \includegraphics[width=\textwidth]{python_data/plots/ti1.pdf}
  \caption{Depolarisationskurve für Heizrate 1.}
  \label{fig:TI1}
\end{figure}

\begin{figure}
  \centering
  \includegraphics[width=\textwidth]{python_data/plots/ti2.pdf}
  \caption{Depolarisationskurve für Heizrate 2.}
  \label{fig:TI2}
\end{figure}

\begin{figure}
  \centering
  \includegraphics[width=\textwidth]{python_data/plots/lin1.pdf}
  \caption{Logarithmierte Stromstärke aufgetragen gegen reziproke Temperatur(Anfangskurve Heizrate1) zur Bestimmung der Aktivierungsenergie.}
  \label{fig:lin1}
\end{figure}

\begin{figure}
  \centering
  \includegraphics[width=\textwidth]{python_data/plots/lin2.pdf}
  \caption{Logarithmierte Stromstärke aufgetragen gegen reziproke Temperatur(Anfangskurve Heizrate2) zur Bestimmung der Aktivierungsenergie.}
  \label{fig:lin2}
\end{figure}

\FloatBarrier
\begin{table}
  \centering
  \caption{Jeweils für den Fit der Korrekturunktion verwendete Messwerte(Indizes an I und T geben die Nummer der Messung an).}
  \label{tab:korrekturfit}
    \begin{tabular}{c c c c}
      \toprule
      $I_{1} \text{ in } \si{\pico\ampere}$ & $ T_{1} \text{ in } \si{\kelvin} $ & $I_{2} \text{ in } \si{\pico\ampere}$ & $ T_{2} \text{ in } \si{\kelvin} $ \\
      \midrule
      20.70 & 239.65 & 13.8 & 228.45\\
      26.70 & 242.25 & 34.0 & 278.65\\
      102.0 & 284.75 & 35.0 & 279.85\\
      105.0 & 286.95 & 35.0 & 281.15\\
            &        & 35.0 & 282.35\\
            &        & 36.0 & 283.45\\
            &        & 36.0 & 284.85\\
            &        & 36.0 & 286.15\\
            &        & 37.0 & 287.45\\
      \bottomrule
    \end{tabular}
\end{table}


\begin{table}
  \centering
  \caption{Korrigierte Messwerte für die Anfangskurve der ersten Messung zur Bestimmung der Aktivierungsenergie.}
  \label{tab:linfit1}
    \begin{tabular}{c c}
      \toprule
      $I \text{ in } \si{\pico\ampere}$ & $ T \text{ in } \si{\kelvin} $ \\
      \midrule
        4,91 & 244,55\\
       12,29 & 247,15\\
       24,67 & 249,75\\
       41,93 & 251,85\\
      106,02 & 254,05\\
      \bottomrule
    \end{tabular}
\end{table}


\begin{table}
  \centering
  \caption{Korrigierte Messwerte für die Anfangskurve der zweiten Messung zur Bestimmung der Aktivierungsenergie.}
  \label{tab:linfit2}
  \begin{tabular}{c c}
    \toprule
    $I \text{ in } \si{\pico\ampere}$ & $ T \text{ in } \si{\kelvin} $ \\
    \midrule
     6,84 & 239,15\\
     9,28 & 240,55\\
    11,81 & 241,75\\
    15,26 & 243,15\\
    17,78 & 244,35\\
    20,55 & 244,95\\
    23,31 & 245,55\\
    25,99 & 246,35\\
    27,80 & 246,85\\
    31,52 & 247,55\\
    34,28 & 248,15\\
    38,04 & 248,75\\
    42,77 & 249,45\\
    47,49 & 250,15\\
    51,25 & 250,75\\
    55,98 & 251,45\\
    61,70 & 252,15\\
    \bottomrule
  \end{tabular}
\end{table}

\FloatBarrier


\subsection{Bestimmung der Aktivierungsenergie aus dem gesamten relevanten Kurvenverlauf}
\label{subsec:method2}

Um die Aktvierungsenergien genauer zu bestimmen zu können, wird im Folgenden jeweils der gesamte Verlauf, der in der Theorie vorhergesagten Depolarisationskurve,
verwendet. Dazu wird Zusammenhang \eqref{eq:t:15} verwendet. Die Integrationen wurden unter \textbf{Python 3} mit der Trapez-Integration aus dem Paket \textbf{scipy.integrate} durchgeführt.
Durch einen Anschließenden linearen Fit der reziproken Temperatur aufgetragen gegen die rechte Seite von Gleichung \eqref{eq:t:15} lässt sich die
Aktivierungsenergie\footnote{In der Einheit Elektronenvolt.} durch
\\
\begin{equation}
  \label{eqn:W2}
  W = s \cdot k_\text{B} / e
\end{equation}
bestimmen.


Es ergeben sich
\begin{align*}
  W_{1} &= \SI{0.88+-0.06}{\electronvolt} \\
  W_{2} &= \SI{0.741+-0.017}{\electronvolt}.
\end{align*}
Dabei wurden die Fit-Parameter\\
\begin{align*}
  s_{1} &= \SI{10200+-700}{\kelvin}  \\
  s_{2} &= \SI{8600+-190}{\kelvin}  \\
  b_{1} &= \SI{-36.3+-2.8}{}  \\
  b_{2} &= \SI{-30.9+-0.8}{}
\end{align*}
berechnet.\\ \\
Die jeweiligen relativen Abweichungen der bestimmten Aktivierungsenergien vom Literaturwert \cite{literaturwert}
\begin{align*}
  W_\text{literatur} = \SI{0.66}{\electronvolt}
\end{align*}
betragen
\begin{align*}
  \delta W_{1} &= \frac{\lvert W_{1} - W_\text{literatur} \rvert}{W_\text{literatur}} &= \SI{34+-10}{\percent}\\
  \delta W_{2} &= \frac{\lvert W_{2} - W_\text{literatur} \rvert}{W_\text{literatur}} &= \SI{12.3+-2.5}{\percent} .
\end{align*}


Bei der Integration wurde nur nach Möglichkeit soweit integriert, bis die Strom-Temperatur-Kurve auf einen Wert möglichst nahe an $0$ abfällt,
Die linearen Fits sowie die jeweils dazu gehörenden umgerechneten Messwerte sind in den Abbildungen \ref{fig:lin1l} und \ref{fig:lin2l} graphisch dargestellt.
Die jeweils verwendeten Messwerte sind in den Tabellen \ref{tab:lnlinfit1} und \ref{tab:lnlinfit2} zu finden.\\

\begin{figure}
  \centering
  \includegraphics[width=\textwidth]{python_data/plots/lnlin1.pdf}
  \caption{Linearer Fit zur bestimmung der Aktivierungsenergie aus dem Gesamtverlauf der Depolarisationskurve(Heizrate 1).}
  \label{fig:lin1l}
\end{figure}

\begin{figure}
  \centering
  \includegraphics[width=\textwidth]{python_data/plots/lnlin2.pdf}
  \caption{Linearer Fit zur bestimmung der Aktivierungsenergie aus dem Gesamtverlauf der Depolarisationskurve(Heizrate 2).}
  \label{fig:lin2l}
\end{figure}

\FloatBarrier

\begin{table}
  \centering
  \caption{Korrigierte Messwerte für die Gesamtkurve der ersten Messung zur Bestimmung der Aktivierungsenergie.}
  \label{tab:lnlinfit1}
  \begin{tabular}{c c}
    \toprule
    $I \text{ in } \si{\pico\ampere}$ & $ T \text{ in } \si{\kelvin} $ \\
    \midrule
      4,91 & 244,55\\
     12,29 & 247,15\\
     24,67 & 249,75\\
     41,93 & 251,85\\
    106,02 & 254,05\\
    134,58 & 256,55\\
    168,96 & 259,15\\
    196,80 & 262,05\\
    204,18 & 264,65\\
    180,85 & 267,65\\
    131,40 & 270,15\\
     79,31 & 272,45\\
     38,51 & 275,15\\
     19,42 & 277,45\\
      9,15 & 279,85\\
      4,53 & 282,45\\
      0,45 & 284,75\\
    \bottomrule
  \end{tabular}
\end{table}
\FloatBarrier



\begin{longtable}{c c}
\caption{Verwendete Messwerte für die Gesamtkurve der zweiten Messung zur Bestimmung der Aktivierungsenergie.}
\label{tab:lnlinfit2}\\
\hline
$I \text{ in } \si{\pico\ampere}$ & $ T \text{ in } \si{\kelvin} $ \\
\hline
\endhead
3,42 & 235,15\\
5,90 & 236,45\\
8,39 & 237,75\\
6,84 & 239,15\\
9,28 & 240,55\\
11,81 & 241,75\\
15,26 & 243,15\\
17,78 & 244,35\\
20,55 & 244,95\\
23,31 & 245,55\\
25,99 & 246,35\\
27,80 & 246,85\\
31,52 & 247,55\\
34,28 & 248,15\\
38,04 & 248,75\\
42,77 & 249,45\\
47,49 & 250,15\\
77,75 & 254,55\\
51,25 & 250,75\\
55,98 & 251,45\\
61,70 & 252,15\\
83,44 & 255,35\\
89,24 & 255,85\\
91,76 & 257,05\\
94,53 & 257,65\\
94,33 & 258,15\\
94,05 & 258,85\\
93,77 & 259,55\\
93,50 & 260,25\\
90,26 & 260,85\\
86,95 & 261,65\\
83,71 & 262,25\\
77,47 & 262,85\\
74,16 & 263,65\\
64,88 & 264,35\\
58,53 & 265,25\\
52,33 & 265,75\\
46,01 & 266,55\\
39,73 & 267,25\\
33,42 & 268,05\\
30,10 & 268,85\\
23,87 & 269,45\\
17,51 & 270,35\\
14,31 & 270,85\\
11,08 & 271.45\\
8,83 & 272,05\\
6,56 & 272,75\\
4,32 & 273,35\\
3,08 & 273,95\\
2,81 & 274,65\\
1,58 & 275,25\\
1,25 & 276,05\\
1,02 & 276,65\\
0,74 & 277,35\\
0,23 & 278,65\\
0,76 & 279,85\\
\hline
\end{longtable}


\subsection{Bestimmung der Relaxationszeit anhand der Position des Maximums der Depolarisationskurve}
\label{subsec:zeit}
Um die Relaxationszeit zu bestimmen wird jeweils die Temperatur $T_\text{max}$ bestimmt, bei der die Depolarisationskurve ihr Maximum annimmt.
Daraus kann dann unter Verwendung des Zusammenhangs \eqref{eq:t:16} und Kenntnis der Aktivierungsenergie, $\tau_{0}$ bestimmt werden.
Für $T_0$ wurde ein Wert von $\SI{210}{\kelvin}$  verwendet.
Es ergeben sich die Werte
\\
\begin{align*}
  \tau_{1} &= \SI{0.5+-1.3}{\femto\second}\\
  \tau_{2} &= \SI{90+-70}{\femto\second}.
\end{align*}

Für die Berechnung der vorausgegangenen Größen wurden die durchschnittlichen
Heizraten
\\
\begin{align*}
  b_{1} &= \SI{0.23}{\kelvin\per\second}\\
  b_{2} &= \SI{0.27}{\kelvin\per\second}
\end{align*}

sowie die Temperaturen $T^\text{max}$ bei denen der maximale Depolarisationsstrom gemessen wurde
\\
\begin{align*}
  T^\text{max}_{1} &= \SI{264.65}{\kelvin}\\
  T^\text{max}_{2} &= \SI{257.65}{\kelvin}
\end{align*}
verwendet.\\
Direkt auffällig ist, dass die Fehler der Relaxationszeiten eine ähnliche
Größe haben wie die berechneten Zeiten an sich.\\ \\
Die jeweiligen relativen Abweichungen der bestimmten Relationszeiten vom Literaturwert \cite{literaturwert}
\begin{align*}
  \tau_{0}^\text{literatur} = \SI{0.66}{\electronvolt}
\end{align*}
betragen
\begin{align*}
  \delta \tau_{1} &= \frac{\lvert \tau_{1} - \tau_{0}^\text{literatur} \rvert}{\tau_{0}^\text{literatur}} &= \SI{98.8+-3.3}{\percent}\\
  \delta \tau_{2} &= \frac{\lvert \tau_{2} - \tau_{0}^\text{literatur} \rvert}{\tau_{0}^\text{literatur}} &= \SI{130.+-180}{\percent} .
\end{align*}
