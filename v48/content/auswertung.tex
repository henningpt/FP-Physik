\section{Auswertung}
\label{sec:Auswertung}

\subsection{Bestimmung der Aktivierungsenergie aus dem anfänglichen Verlauf der Depolarisationskurve}
\label{subsec:method1}
Da sich der Zusammenhang für die Depolarisationskurve zu Beginn durch die Gleichung \eqref{eqn:anfangdepolar} in guter Näherung
beschreiben lässt, ergibt sich für den logarithmierten Depolarisationsstrom in Abhängigkeit von $T^{-1}$ der Zusammenhang
\\
\begin{equation}
  \label{eqn:lineardepolar}
  \ln{I} = \frac{p^2 \cdot E N_\text{p} }{3 \cdot k_{b}
   \cdot T_\text{p} \cdot \tau_{0}} -\frac{W}{k_{\text{B}}} \cdot T^{-1}
\end{equation}

Aus der Steigung des linearen Zusammenhangs \eqref{eqn:lineardepolar} lässt sich erkennen, dass sich
die Aktivierungsenergie $W$\footnote{In Elektronenvolt} mit Hilfe eines linearen Fits der ersten Messwerte über
\\
\begin{equation}
  \label{eqn:W}
  W = -s \cdot k_\text{B} / e
\end{equation}
bestimmen lässt. Dabei bezeichnet $s$ die Steigung Fit-Funktion, $k_\text{B}$ die Boltzmann-Konstante und $e$ die Elementarladung.

Anhand der in \ref{tab:linfit1} beziehungweise \ref{tab:linfit2} Wertepaare ergeben sich über einen unter \textbf{Python 3}
durchgeführten linearen Fit, der mit der Funktion \textbf{curveFit} aus
dem Paket \textbf{scipy} durchgeführt wurde, die Energien
\\
\begin{align*}
  W_{1} &= \SI{0.262+-0.030}{\electronvolt} \\
  W_{2} &= \SI{0.248+-0.017}{\electronvolt}.
\end{align*}
Dabei wurden die Fit-Parameter
\\
\begin{align*}
  s_{1} &= \SI{-3040+-350}{}\\
  s_{2} &= \SI{-2870+-200}{}\\
  b_{1} &= \SI{-0.8+-1.5}{}\\
  b_{2} &= \SI{-0.3+-0.9}{}
\end{align*}
berechnet.
Die Fehler wurden anhand der von \textbf{curveFit} berechneten Kovarianzmatrix bestimmt. Alle Fehlerfortpflanzungen in diesem Versuchsprotokoll
werden mit dem Paket \textbf{uncertainties}\cite{uncertainties} durchgeführt.

Die Aufgenommenen Depolarisationskurven sind in Abbildungen \ref{fig:TI1} und \ref{fig:TI2} gezeigt. Da sich zusätzlich zum ersten Maximum noch weitere anschließen und die
Kurve nicht wie im theoretischen Idealfall auf $0$ abfällt, wurde zur Vereinfachung der Auswertung, wie in \ref{fig:TI1} und \ref{fig:TI2} dargestellt jeweils eine lineare Korrekturfunktion
im problematischen Bereich abgezogen. Dabei wurden weiterhin die Werte die anschließend unterhalb von $0$ lagen nach oben korrigiert.
Graphische Darstellungen der linearen Fits\footnote{Fits zur Bestimmung der Aktivierungsenergien.} durch die dafür verwendeten Messwerte
sind in Abbildungen \ref{fig:lin1} und \ref{fig:lin2} gegeben.

\begin{figure}
  \centering
  \includegraphics[width=\textwidth]{python_data/plots/ti1.pdf}
  \caption{Depolarisationskurve für Heizrate 1.}
  \label{fig:TI1}
\end{figure}

\begin{figure}
  \centering
  \includegraphics[width=\textwidth]{python_data/plots/ti2.pdf}
  \caption{Depolarisationskurve für Heizrate 2.}
  \label{fig:TI2}
\end{figure}

\begin{figure}
  \centering
  \includegraphics[width=\textwidth]{python_data/plots/lin1.pdf}
  \caption{Logarithmierte Stromdichte aufgetragen gegen reziproke Temperatur(Anfangskurve Heizrate1) zur Bestimmung der Aktivierungsenergie.}
  \label{fig:lin1}
\end{figure}

\begin{figure}
  \centering
  \includegraphics[width=\textwidth]{python_data/plots/lin2.pdf}
  \caption{Logarithmierte Stromdichte aufgetragen gegen reziproke Temperatur(Anfangskurve Heizrate2) zur Bestimmung der Aktivierungsenergie.}
  \label{fig:lin2}
\end{figure}

\FloatBarrier
\begin{table}
  \centering
  \caption{Verwendete Messwerte für die Anfangskurve der ersten Messung zur Bestimmung der Aktivierungsenergie.}
  \label{tab:linfit1}
    \begin{tabular}{c c}
      \toprule
      $I \text{ in } \si{\pico\ampere}$ & $ T \text{ in } \si{\kelvin} $ \\
      \midrule
       7.90 & 211.35\\
       7.80 & 214.85\\
       8.10 & 218.05\\
       8.70 & 220.75\\
       9.30 & 223.65\\
       9.90 & 226.45\\
      11.40 & 229.55\\
      12.00 & 232.15\\
      14.40 & 234.75\\
      15.30 & 237.15\\
      20.70 & 239.65\\
      26.70 & 242.25\\
      35.00 & 244.55\\
      47.00 & 247.15\\
      64.00 & 249.75\\
      85.00 & 251.85\\
      \bottomrule
    \end{tabular}
\end{table}


\begin{table}
  \centering
  \caption{Verwendete Messwerte für die Anfangskurve der zweiten Messung zur Bestimmung der Aktivierungsenergie.}
  \label{tab:linfit2}
  \begin{tabular}{c c}
    \toprule
    $I \text{ in } \si{\pico\ampere}$ & $ T \text{ in } \si{\kelvin} $ \\
    \midrule
    10.50 & 213.15\\
    10.80 & 214.15\\
    10.20 & 215.85\\
     9.60 & 217.65\\
     8.40 & 219.35\\
    22.20 & 222.55\\
    16.50 & 223.95\\
    16.50 & 225.55\\
    14.40 & 226.95\\
    13.80 & 228.45\\
    13.50 & 230.05\\
    13.50 & 231.15\\
    13.80 & 232.45\\
    24.00 & 233.75\\
    20.00 & 235.15\\
    23.00 & 236.45\\
    26.00 & 237.75\\
    25.00 & 239.15\\
    28.00 & 240.55\\
    31.00 & 241.75\\
    35.00 & 243.15\\
    38.00 & 244.35\\
    41.00 & 244.95\\
    45.00 & 245.55\\
    47.00 & 246.35\\
    49.00 & 246.85\\
    53.00 & 247.55\\
    56.00 & 248.15\\
    60.00 & 248.75\\
    65.00 & 249.45\\
    70.00 & 250.15\\
    74.00 & 250.75\\
    79.00 & 251.45\\
    \bottomrule
  \end{tabular}
\end{table}

\FloatBarrier


\subsection{Bestimmung der Aktivierungsenergie aus dem gesamten relevanten Kurvenverlauf}
\label{subsec:method2}

Um die Aktvierungsenergien genauer zu bestimmen zu können, wird im Folgenden jeweils der gesamte Verlauf, der in der Theorie vorhergesagten Depolarisationskurve,
verwendet. Dazu wird Zusammenhang \eqref{eqn:depolar} verwendet. Die Integrationen wurden unter \textbf{Python 3} mit der Simpson-Integration aus dem Paket \textbf{scipy.integrate} durchgeführt.
Durch einen Anschließenden linearen Fit der reziproken Temperatur aufgetragen gegen die rechte Seite von Gleichung \eqref{eqn:depolar} lässt sich die
Aktivierungsenergie\footnote{In der Einheit Elektronenvolt.} durch
\\
\begin{equation}
  \label{eqn:W2}
  W = s \cdot k_\text{B} / e
\end{equation}
bestimmen.


Es ergeben sich
\\
\begin{align*}
  W_{1} &= \SI{0.22+-0.05}{\electronvolt} \\
  W_{2} &= \SI{0.205+-035}{\electronvolt}.
\end{align*}
Dabei wurden die Fit-Parameter
\\
\begin{align*}
  s_{1} &= \SI{2600+-600}{}  \\
  s_{2} &= \SI{2400+-400}{}  \\
  b_{1} &= \SI{-7.8+-2.5}{}  \\
  b_{2} &= \SI{-4.8+-1.6}{}
\end{align*}
berechnet.
\\
Bei der Integration wurde nur nach möglichkeit soweit integriert, bis die Strom-Temperatur-Kurve auf einen Wert nahe des Werts zu Beginn der Integration abfällt, dies war vor allem für
die Werte der zweiten Messung nur bedingt realisierbar.
Die linearen Fits sowie die jeweils dazu gehörenden umgerechneten Messwerte sind in den Abbildungen \ref{fig:lin1l} und \ref{fig:lin2l} graphisch dargestellt.
Die jeweils verwendeten Messwerte sind in den Tabellen \ref{tab:lnlinfit1} und \ref{tab:lnlinfit2} zu finden.
\\

\begin{figure}
  \centering
  \includegraphics[width=\textwidth]{python_data/plots/lnlin1.pdf}
  \caption{Linearer Fit zur bestimmung der Aktivierungsenergie aus dem Gesamtverlauf der Depolarisationskurve(Heizrate 1).}
  \label{fig:lin1l}
\end{figure}

\begin{figure}
  \centering
  \includegraphics[width=\textwidth]{python_data/plots/lnlin2.pdf}
  \caption{Linearer Fit zur bestimmung der Aktivierungsenergie aus dem Gesamtverlauf der Depolarisationskurve(Heizrate 2).}
  \label{fig:lin2l}
\end{figure}

\FloatBarrier

\begin{table}
  \centering
  \caption{Verwendete Messwerte für die Gesamtkurve der ersten Messung zur Bestimmung der Aktivierungsenergie.}
  \label{tab:lnlinfit1}
  \begin{tabular}{c c}
    \toprule
    $I \text{ in } \si{\pico\ampere}$ & $ T \text{ in } \si{\kelvin} $ \\
    \midrule
      9.30 & 223.65\\
      9.90 & 226.45\\
     11.40 & 229.55\\
     12.00 & 232.15\\
     14.40 & 234.75\\
     15.23 & 237.15\\
     20.70 & 239.65\\
     26.70 & 242.25\\
     35.00 & 244.55\\
     47.00 & 247.15\\
     64.00 & 249.75\\
     85.00 & 251.85\\
    152.99 & 254.05\\
    186.00 & 256.55\\
    224.99 & 259.15\\
    257.99 & 262.05\\
    269.99 & 264.65\\
    252.00 & 267.65\\
    207.00 & 270.15\\
    158.99 & 272.45\\
    122.99 & 275.15\\
     44.34 & 277.45\\
     35.39 & 279.85\\
     32.19 & 282.45\\
     29.37 & 284.75\\
    \bottomrule
  \end{tabular}
\end{table}

\begin{table}
  \centering
  \caption{Verwendete Messwerte für die Gesamtkurve der zweiten Messung zur Bestimmung der Aktivierungsenergie.}
  \label{tab:lnlinfit2}
  \begin{tabular}{c c}
    \toprule
    $I \text{ in } \si{\pico\ampere}$ & $ T \text{ in } \si{\kelvin} $ \\
    \midrule
     28.00 & 267.25\\
     31.00 & 268.05\\
     35.00 & 268.85\\
     38.00 & 269.45\\
     41.00 & 213.15\\
     45.00 & 214.15\\
     47.00 & 215.85\\
     49.00 & 217.65\\
     53.00 & 219.35\\
     56.00 & 222.55\\
     60.00 & 223.95\\
     65.00 & 225.55\\
     70.00 & 226.95\\
     75.00 & 228.45\\
     79.00 & 230.05\\
     85.00 & 231.15\\
    102.00 & 232.45\\
    109.00 & 233.75\\
    114.00 & 347.75\\
    117.00 & 349.05\\
    120.00 & 350.35\\
    120.00 & 351.85\\
    120.00 & 353.35\\
    120.00 & 354.85\\
    120.00 & 356.45\\
    117.00 & 358.25\\
    114.00 & 359.45\\
    111.00 & 360.75\\
    105.00 & 235.15\\
    102.00 & 236.45\\
     93.00 & 237.75\\
     87.00 & 239.15\\
     81.00 & 240.55\\
     75.00 & 241.75\\
     69.00 & 243.15\\
     63.00 & 244.35\\
     60.00 & 244.95\\
    \bottomrule
  \end{tabular}
\end{table}
\FloatBarrier

\subsection{Bestimmung der Relaxationszeit anhand der lage des Maximums der Depolarisationskurve}
\label{subsec:zeit}
Um die Relaxationszeit zu bestimmen wird jeweils die Temperatur $T_\text{max}$ bestimmt, bei der die Depolarisationskurve ihr Maximum annimmt.
Daraus kann dann unter Verwendung des Zusammenhangs \eqref{eqn:relaxzeit} und Kenntnis der Aktivierungsenergie, $\tau_{0}$ bestimmt werden.
Es ergeben sich die Werte
\\
\begin{align*}
  \tau_{01} &= \SI{0.16+-0.25}{\micro\second}\\
  \tau_{02} &= \SI{2.6+-1.9}{\micro\second}.
\end{align*}

Für die Berechnung der vorausgegangenen Größen wurden die durchschnittlichen
Heizraten
\\
\begin{align*}
  b_{1} = \SI{0.04}{\kelvin\per\second}\\
  b_{2} = \SI{0.02}{\kelvin\per\second}
\end{align*}

sowie die Temperaturen $T^\text{max}$ bei denen der maximale Depolarisationsstrom gemessen wurde
\\
\begin{align*}
  T^\text{max}_{1} = \SI{264.65}{\kelvin}
  T^\text{max}_{2} = \SI{257.65}{\kelvin}
\end{align*}
verwendet.

Direkt auffällig ist, dass die Fehler der Relaxationszeiten eine ähnliche
Größe haben wie die berechneten Zeiten an sich.
