\section{Theorie}
\label{sec:Theorie}
Bei den Temperaturen, die bei den folgenden Untersuchungen erreicht werden, bewegen sich im Kristall hauptsächlich die zweiwertigen Ionen.
Um eine neue Position im Gitter einzunehmen, muss eine Potential Barriere überwunden werden.
Die dadurch bestimmte Energie ist die Aktivierungsenergie $W$.
Aus statistischen Überlegungen lässt sich der Anteil der Ionen, die diese Energie besitzen, als 
\begin{align}
\exp \left(\frac{-W}{kT}\right)
\label{eq:1}
\end{align}
bestimmen.
Ebenso folgt die Relaxationszeit
\begin{align}
\tau(T)=\tau_0 \exp (\frac{W}{kT}\text{ ,}
\label{eq:2}
\end{align}
welche die mittlere Zeit zwischen zwei Positionswechseln beschreibt.

\noindent
Durch Anlegen eines elektrischen Feldes lassen sich die Dipole in eine Vorzugsrichtung ausrichten.
Dabei ist zu beachten, dass aufgrund der thermischen Bewegungen nur ein bestimmter Anteil $y$ aller Dipole in diese Richtung weist.
Dieser Anteil ist durch die Langevin-Funktion 
\begin{align}
y=L(x)=\cot (x) -\frac{1}{x}
\label{eq:3}
\end{align}
gegeben, wobei $x=\frac{pE}{kT}$ gewählt wird.
Da unter den folgenden Messbedingungen $pE<<kT$ gilt kann $y$ durch
\begin{align}
y(T)=\frac{pE}{3kT}
\label{eq:4}
\end{align}
genährt werden.
Die hier gewählte Messmethode erfordert ein Abkühlen der Probe, um den durch das elektrische Feld erzeugten Zustand einzufrieren.
Durch Aufheizen der Probe können mit den im folgenden beschriebenen Methoden die Aktivierungsenergie $W$ und die Konstante $\tau_0$ aus Formel (\ref{eq:2}) bestimmt werden. 
Um die Rechnung zu vereinfachen wird eine konstante Heizrate
\begin{align}
b=\frac{dT}{dt}=\text{const}
\label{eq:5}
\end{align}
angenommen.
Das Aufheizen der Probe und die damit verknüpfte wachsende thermische Energie des Kristallgitters führt zu einer wachsenden Anzahl an Dipolen, die aus der Vorzugsrichtung herausspringen können.
In einem externen Stromkreis kann daher ein Depolarisationsstrom gemessen werden. 
Die Depolarisationsstromdichte $j$ hängt von der Gesamtpolarisation, der Anzahl der Dipole in Vorzugsrichtung bei der Polarisationstemperatur $T_p$ und der Änderung der Dipole in Vorzugsrichtung ab
\begin{align}
j(T)=y(T_p)~p~\frac{dN}{dt} \overset{Gl. (\ref{eq:4})}{=} \frac{p^2E}{rkT_p}\frac{dN}{dt}\text{ .}
\label{eq:6}
\end{align}
Die Anzahl der relaxierenden Dipole ist proportional zur Anzahl der ausgerichteten Dipole
\begin{align}
\frac{dN}{dt}=-\frac{N}{\tau(T)} \text{ .}
\end{align}
Daraus ergibt sich unter dem Einfluss der konstanten Heizrate
\begin{align}
N=N_p\exp \left(-\frac{1}{b} \int_{T_0}^T \frac{dT\prime}{\tau(T\prime)}\right) \text{ .}
\end{align}
Zusammen mit den Gleichungen (\ref{eq:4}) und (\ref{eq:6}) folgt für die Depolarisationsstromdichte
\begin{align}
j(T)=\frac{p^2E}{3kT}\frac{N_p}{\tau_0}\exp \left(-\frac{W}{kT}\right)\exp\left(-\frac{1}{b\tau_0}\int_{T_0}^T\exp\left(-\frac{W}{kT}\right)dT\prime\right) \text{ .}
\end{align}
