\section{Diskussion}
\label{sec:Diskussion}
Bei der Bestimmung Aktivierungsenergie $W$ anhand des Beginns der Depolarisationskurve,
ergeben sich mit $\SI{1.66+-0.09}{\electronvolt}$ und $\SI{0.870+-0.011}{\electronvolt}$,
anzeigte, für die beiden Heizraten sehr unterschiedliche Werte. Dabei weichen die Werte um \SI{152+-13}{\percent} beziehungsweise \SI{31.9+-1.7}{\percent} ab.
Während sich bei Abweichung der zweiten Messung noch von einem mit der literatur kompatiblen Ergebnis sprechen lässt, ist dies für die erste Messung nicht
gegeben. Dies kann dadurch erklärt werden, dass bei der ersten Messung die exponentielle Anfangskurve deutlich weniger fein ausgemessen wurde.\\ \\
Die Bestimmung der Aktivierungsenergie über den gesamten Kurvenverlauf lieferte mit
$\SI{0.88+-0.06}{\electronvolt}$ und $\SI{0.741+-0.017}{\electronvolt}$ deutlich ähnlichere Werte als bei der ersten Methode.
Die relativen Abweichungen von \SI{34+-10}{\percent} und \SI{12.3+-2.5}{\percent} sind für beide Messungen jeweils geringer.
Vor allem bei der ersten Messung ist ein deutlicher unterschied zu beobachten, was damit zu tun hat das im Vergleich mit der Anfangskurve, der übrige
Kurvenverlauf feiner ausgemessen wurde. Dass die sich für die zweite Messung wieder ein Wert näher am Literaturwert ergibt kann einerseits wieder
dadurch begründet werden, dass insgesamt mehr Messwerte für den Kurvenverlauf auf genommen wurden und andererseits dadurch dass die Heizrate vor allem
zum Ende des relevanten Kurvenverlaufs besser konstant gehalten werden konnte.\\
Da im hier beschriebenen Experiment Werte ähnlich dem Literaturwert gefunden werden konnten, konnte die Größenordnung der Aktivierungsenergie
für ,mit Strontium dotiertem Kalium-Bromid, von mehreren Zehnteln Elektronenvolt bestätigt werden.\\ \\
Die experimentell ermittelten Relaxationszeiten $\SI{0.5+-1.3}{\femto\second}$ und $\SI{90+-70}{\femto\second}$ weisen eine große Differenz auf,
was darauf zurückzuführen sein kann, dass bei der ersten Messung die Heizrate am Anfang zu hoch angesetzt wurde und zum Ende nicht mehr konstant
gehalten werden konnte. Eine Gemeinsamkeit der beiden Zeiten ist allerdings der sehr große Fehler.
Dieser kommt vor allem dadurch zustande, dass der Fehler der Aktivierungsenergie sich jeweils in der Funktion \eqref{eq:t:16} fortpflanzt.
Durch die sehr großen Fehler haben die bestimmten Relaxationszeiten nur eine sehr geringe Aussagekraft. Allerdings lässt sich feststellen,
dass für den zweiten Wert der Literaturwert innerhalb der angegebenen Unsicherheit liegt.\\ \\
Während die Relaxationszeiten kein sinnvolles Ergebnis liefern, lässt sich über die Aktvierungsenergie und deren Korrektheit noch diskutieren.
Um die berechneten Abweichunge zu erklären, lassen sich die Folgenden Einflüsse nennen, die das Ergebnis dieses Experiments verfälscht haben.
Dabei sind die hohe Empfindlichkeit des verwendeten Amperemeters, die Schwierigkeit eine konstante Heizrate zu gewährleisten und Ablesefehler zu nennen.
Weiterhin wurde wie schon erwähnt das Amperemeter bei der zweiten Messung erst zu Messbeginn in Betrieb genommen, wodurch die ersten aufgenommenen Werte
unbrauchbar wurden.\\
Ebenso ist möglicherweise die Tatsache, dass die beiden Messreihen an unterschiedlichen Tagen und Zeiten aufgenommen wurden,
sodass beispielsweise die unterschiedlichen Wetterverhältnisse
die Messungen jeweils anders beeinflusst haben.
