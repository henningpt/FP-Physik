\section{Diskussion}
\label{sec:Diskussion}
Bei der Bestimmung Aktivierungsenergie $W$ anhand des Beginns der Depolarisationskurve, ergeben sich mit $\SI{0.262+-0.030}{\electronvolt}$ und $\SI{0.248+-0.017}{\electronvolt}$, obwohl das Amperemeter
aufgrund der Tatsache, dass es bis kurz vor Start der zweiten Messung ausgeschaltet war, zu Beginn der zweiten Messung verschiedene sehr unterschiedliche Werte(siehe Abbildung \ref{fig:TI2})
anzeigte, für die beiden Heizraten sehr ähnliche Werte. Möglicherweise wurde die besagte Ungenauigkeit bei der zweiten Messung dadurch ausgeglichen, dass der Fit an einem größeren Teil der Kurve
vorgenommen wurde.


Die Bestimmung der Aktivierungsenergie über den gesamten Kurvenverlauf lieferte mit $\SI{0.220+-0.035}{\electronvolt}$ und $\SI{0.205+-0.35}{\electronvolt}$ ebenfalls sehr ähnliche Werte
die allerdings beide geringer waren als bei der ersten Bestimmungsmethode. Problematisch hierbei war vor allem die Integration des Stroms, da das Integral je nachdem über welches Intervall integriert
wurde zum Teil nicht bestimmbar wurde. Dies fiel, wie schon in der Auswertung erwähnt, vor allem bei der zweiten Heizrate ins Gewicht, da hier bei weitem nicht bis zum Kurvenende integriert werden konnte.


Da alle ermittelten Aktivierungsenergien ähnliche Werte liefern, ist davon auszugehen, dass der tatsächliche Wert zumindest in der gleichen Größenordnung wie $\SI{0.2}{\electronvolt}$ liegt.


Die experimentell ermittelten Relaxationszeiten $\SI{0.16+-0.25}{\micro\second}$ und $\SI{2.6+-1.9}{\micro\second}$ weisen eine große Differenz auf, was darauf zurückzuführen sein kann, dass
bei der ersten Messung die Heizrate am Anfang zu hoch angesetzt wurde und zum Ende nicht mehr konstant gehalten werden konnte. Eine Gemeinsamkeit der beiden Zeiten ist allerdings der sehr große
Fehler. Dieser kommt vor allem dadurch zustande, dass der Fehler der Aktivierungsenergie sich jeweils in der Funktion \eqref{eq:t:16} fortpflanzt. Durch die sehr großen Fehler haben die bestimmten
Relaxationszeiten keine Aussagekraft.


Während die Relaxationszeiten kein sinnvolles Ergebnis liefern, lässt sich über die Aktvierungsenergie und deren Korrektheit noch diskutieren.
Zwar ist es aufgrund eines fehlenden Referenzwerts schwierig einzuschätzen ob Energien nahe am wahren Wert gefunden wurden, lassen sich Einflüsse nennen
die das Ergebnis dieses Experiments verfälscht haben.
Dabei sind die hohe Empfindlichkeit des verwendeten Amperemeters, die Schwierigkeit eine konstante Heizrate zu gewährleisten und Ablesefehler zu nennen.
Weiterhin wurde wie schon erwähnt das Amperemeter bei der zweiten Messung erst zu Messbeginn in Betrieb genommen, was die ersten aufgenommenen Werte verfälscht hat.
Ebenso ist möglicherweise die Tatsache, dass die beiden Messreihen an unterschiedlichen Tagen und Zeiten aufgenommen wurden, sodass beispielsweise die unterschiedlichen Wetterverhältnisse
die Messungen jeweils anders beeinflusst haben.
