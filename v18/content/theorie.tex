\section{Theorie}
\label{sec:Theorie}


\subsection{Gamme-Strahlung in Materie}
\label{subsec:t1}

In dem hier thematisierten Versuch wird von einer Probe ausgesandte Gamma-Strahlung untersucht.
Diese Strahlung kann auf verschiedene Weise mit den Elektronen sowie Atomkernen im Detektor
interagieren.
Dabei sind vor allem der Photoeffekt, der Comptoneffekt und die Elektron-Positron-
Paarbildung relevant.
Jeder Wechselwirkung lässt sich ein sogenannter Wirkungsquerschnitt
$\sigma$ zuordnen. Dieser ist ein Maß für die Wahrscheinlichkeit,
dass der jeweilige Prozess auftritt. Weiterhin lässt sich damit \\ \\
Beim Photoeffekt gibt das Photon seine Energie vollständig an ein Hüllenelektron ab, was bei
einer Photonenergie oberhalb der Bindungsenergie des Elektrons dafür sorgt, dass das Elektron
aus seinem Energiezustand entfernt werden kann. Überschüssige Energie bleibt als kinetische
Energie des ausgelösten Elektrons erhalten. Die Unterbesetzung der Schale, der das ausgelöste
Elektron enstammte, wird ausgeglichen, indem ein Elektron aus einer energetisch höheren Schale
die unvollständige Schale auffüllt. Dabei gibt das nachrückende Elektron seine Energie wieder
in Form von Röntgenstrahlung ab. \\ \\
Der Comptoneffekt beschreibt die inelastische Streuung hochenergetischer Photonen mit einem
ruhenden Elektron.
Dabei wird ein vom Streuwinkel abhängiger Energieanteil an das Elektron übertragen.
Der maximale Energieübertrag findet bei der Rückstreuung also einem Streuwinkel
von $\SI{180}{\degree}$ statt und es wird die Energie

\begin{equation}
  \label{eqn:e1t1}
  E_\text{l,max} = E_{\gamma} \cdot \frac{2\varepsilon}{1 + 2\varepsilon}
\end{equation}
übertragen. Dabei bezeichnet $\varepsilon$ den Anteil der ursprünglichen
Photonenergie $E_{\gamma}$ an der Ruheenergie des Elektrons.

\subsection{Wirkungsweise}
\label{subsec:Wirkungsweise}
Das wichtigste Element des Germaniumdetektors ist eine Halbleiterdiode.
Die durch p- bzw. n-Dotierung entstehenden Ladungsträger diffudieren und rekombinieren an der Grenzfläche zwischen den beiden Bereichen der Diode. 
Dadurch entsteht um die Grenzfläche herum eine Ladungsarme Zone.
Diese Zone ist für die Messung von $\gamma$-Quanten der entscheidende Bereich der Diode.
Erzeugt ein $\gamma$-Quant innerhalb der Diode ein freies Elektron, zum Beispiel indem mithilfe des Photoeffekts ein Elektron aus dem Valenz- in das Leitungsband gehoben wird, so hebt dieses freie Elektron durch Abgabe seiner kinetischen Energie weitere Elektronen in das Leitungsband.
Entsteht so ein Elektronenschauer in dem positiv oder negativ geladenen Bereich der Diode, wird er durch Rekombination mit den jeweiligen Ladungsträgern des Bereichs eliminiert, und ist damit nicht messbar.
Falls der Elektronenschauer in der Ladungsarmen Zone entsteht, gelingt es durch ein hinreichend großes elektrisches Feld, die freien Elektronen von den zurückgelassenen positiv geladenen Atomrümpfen zu trennen. 
Es entsteht ein Strom $\frac{dQ}{dt}$, welcher nun gemessen werden kann.
Die Gesamtladung $Q$ ist dabei proportional zur Energie des $\gamma$-Quants, welches den Elektronenschauer verursacht hat, da sie ihrerseits proportional zur Anzahl der Elektronen des Schauers ist.
Ein Ziel bei der Konstruktion eines Germaniumdetektors ist es also, eine möglichst breite Ladungsarme Zone zu erreichen.
Dies ist zum Beispiel durch ein starkes elektrisches Feld und eine unsymmetische Dotierung möglich.
Da ein starkes elektrisches Feld den Einfluss thermischer Ströme auf die Messung vergrößert, wird die Diode in der Regel auf niedrige Temperaturen gekühlt, sodass dieser Einfluss verringert wird.


\subsection{Eletronische Beschaltung eines Germanium Detektors}
\label{subsec:Eletronische Beschaltung eines Germanium Detektors}


\cite{sample}
