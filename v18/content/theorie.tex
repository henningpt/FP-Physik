\section{Theorie}
\label{sec:Theorie}


\subsection{Gamme-Strahlung in Materie}
\label{subsec:t1}

In dem hier thematisierten Versuch wird von einer Probe ausgesandte Gamma-Strahlung untersucht.
Diese Strahlung kann auf verschiedene Weise mit den Elektronen sowie Atomkernen im Detektor
interagieren.
Dabei sind vor allem der Photoeffekt, der Comptoneffekt und die Elektron-Positron-
Paarbildung relevant.
jeder Wechselwirkung lässt sich ein sogenannter Wirkungsquerschnitt
$\sigma$ zuordnen. Dieser ist ein Maß für die Wahrscheinlichkeit,
dass der jeweilige Prozess auftritt. Weiterhin lässt sich damit \\ \\
Beim Photoeffekt gibt das Photon seine Energie vollständig an ein Hüllenelektron ab, was bei
einer Photonenergie oberhalb der Bindungsenergie des Elektrons dafür sorgt, dass das Elektron
aus seinem Energiezustand entfernt werden kann. Überschüssige Energie bleibt als kinetische
Energie des ausgelösten Elektrons erhalten. Die Unterbesetzung der Schale, der das ausgelöste
Elektron enstammte, wird ausgeglichen, indem ein Elektron aus einer energetisch höheren Schale
die unvollständige Schale auffüllt. Dabei gibt das nachrückende Elektron seine Energie wieder
in Form von Röntgenstrahlung ab. \\ \\
Der Comptoneffekt beschreibt die inelastische Streuung hochenergetischer Photonen mit einem
ruhenden Elektron.
Dabei wird ein vom Streuwinkel abhängiger Energieanteil an das Elektron übertragen.
Der maximale Energieübertrag findet bei der Rückstreuung also einem Streuwinkel
von $\SI{180}{\degree}$ statt und es wird die Energie

\begin{equation}
  \label{eqn:e1t1}
  E_\text{l,max} = E_{\gamma} \cdot \frac{2\varepsilon}{1 + 2\varepsilon}
\end{equation}
übertragen. Dabei bezeichnet $\varepsilon$ den Anteil der ursprünglichen
Photonenergie $E_{\gamme}$ an der Ruheenergie des Elektrons.

\cite{sample}
