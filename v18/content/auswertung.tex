\section{Auswertung}
\label{sec:Auswertung}

\subsection{Kallibration und Bestimmung der Effizienz}
\label{subsec:a1}
Um das Energiespektrum für unbekannte Strahler sinnvoll interpretieren zu können,
ist eine Kallibration der Energieskala notwendig. Dazu werden die in Tabelle \ref{tab:atab1}
angegebenen Messwerte des Spektrum eines Europium 152 Strahlers verwendet.
Über eine lineare Ausgleichsrechnung, ergeben sich die Fit-Parameter

\begin{align}
  s = \SI{+-}{}
  b = \SI{+-}{}
\end{align}

Um die Effizienz des Detektors zu messen wird Gleichung \eqref{eqn:} verwendet.
Die Aktivität des Europiums betrug am 01.10.2000 \cite{sample}

\begin{align*}
  A = \SI{}{\becquerel},
\end{align*}
bei einer Halbwertszeit von
\begin{align*}
  \tau = \SI{4943+-5}{\day}
\end{align*}
\subsection{Bestimmungen von Detektoreigenschaften}
\label{subsec:a2}






\subsection{Aktivitätsbestimmung von }
\label{subsec:a3}

\subsection{Untersuchung von Zerfallsketten in }
\label{subsec:a4}



\begin{figure}
  \centering
  \includegraphics{plot.pdf}
  \caption{Plot.}
  \label{fig:plot}
\end{figure}
