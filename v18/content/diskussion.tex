\section{Diskussion}
\label{sec:Diskussion}
%(a) Kalibrierung und Effizienzmessung
Die Kalibrierung des Detektors konnte wie aus Abbildung \ref{fig:Kalibrierung} hervorgeht mit hoher Genauigkeit durchgeführt werden.
Damit hat sich auch das in Abschnitt \ref{subsec:a0} beschriebene Verfahren zur Ermittlung der Lagen der Spektrallinien bewährt.
Für die Festlegung der Effizienz in Abhängigkeit der Energie wird darüber hinaus der Inhalt der Peaks benötigt.
Vor allem bei Spektrallinien mit geringer Emissionswahrscheinlichkeit hat \textit{curve\_fit} Probleme mit der Regression. 
Das liegt daran, dass sich solche Peaks kaum vom Untergrund absetzen.
Es wurde versucht diesem Problem durch Variation des Kanalbereichs, über dem die Regression durchgeführt wird, entgegen zu treten.
Teilweise lassen sich so bessere Ergebnisse erzielen, teilweise ist eine Verbesserung nicht möglich.
Vor allem für solche Spektrallinien mit geringerer Emissionswahrscheinlichkeit ist das in Abschnitt \ref{subsec:a0} beschriebene Verfahren nicht ausgereift genug, um die Inhalte der Peaks präzise zu bestimmen.
Insgesamt führt dies zu stark fehlerbehafteten Parametern der Regression.

%(d) Aktivitätsbestimmung von Barium 133
Tabelle \ref{tab:a_d_1} zeigt, dass die berechneten Aktivitäten der Probe starke Unsicherheiten aufweisen.
Der wesentliche Grund dafür sind die großen Unsicherheiten bei der Effizienzmessung, die sich in diese Ergebnisse fortpflanzen.
Ein genauer Wert für die Aktivität kann daher nicht angegeben werden. 
Die Messung liefert lediglich einen Orientierungswert.
Auch mit einer präziseren Effizienzmessung würden die gesammelten Messwerte teils stark voneinander abweichen.
