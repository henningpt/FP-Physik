\section{Diskussion}
\label{sec:Diskussion}
%(a) Kalibrierung und Effizienzmessung
Die Kalibrierung des Detektors konnte wie aus Abbildung \ref{fig:Kalibrierung} hervorgeht mit hoher Genauigkeit durchgeführt werden.
Damit hat sich auch das in Abschnitt \ref{subsec:a0} beschriebene Verfahren zur Ermittlung der Lagen der Spektrallinien bewährt.
Für die Festlegung der Effizienz in Abhängigkeit der Energie wird darüber hinaus der Inhalt der Peaks benötigt.
Vor allem bei Spektrallinien mit geringer Emissionswahrscheinlichkeit hat \textit{curve\_fit} Probleme mit der Regression.
Das liegt daran, dass sich solche Peaks kaum vom Untergrund absetzen.
Es wurde versucht diesem Problem durch Variation des Kanalbereichs, über dem die Regression durchgeführt wird, entgegen zu treten.
Teilweise lassen sich so bessere Ergebnisse erzielen, teilweise ist eine Verbesserung nicht möglich.
Vor allem für solche Spektrallinien mit geringerer Emissionswahrscheinlichkeit ist das in Abschnitt \ref{subsec:a0} beschriebene Verfahren nicht ausgereift genug, um die Inhalte der Peaks präzise zu bestimmen.
Insgesamt führt dies zu stark fehlerbehafteten Parametern der Regression.\\ \\
% b)
Bei der Untersuchung der Detektoreigenschaften konnten mit geringen Unsicherheiten die Lage des Photopeaks,
sowie dessen Halb- und Zehntelwertsbreite bestimmt werden. Das Verhältnis der Teilwertsbreiten entspricht
dem einer Gaussfunktion. Die relative Abweichung der Halbwertsbreite
zum Vergleichswert beträgt $\SI{101.8}{\percent}$. Die numerisch bestimmte Halbwertsbreite ist also doppelt groß wie erwartet, sodass nahe liegt, dass zum
gemessenen Photopeak auch andere Effekte beitragen. Bei der Analyse des Compton-Kontinuums konnte eine sehr
geringe Abweichung der abgelesenen Werte von den aus der Gammaenergie berechneten Werte
für Rückstreupeak und Comptonkante festgestellt werden. Hingegen stimmt das Verhältnis der bestimmten Inhalte($1.92+-0.08$) von
Kontinuum und Photopeak kaum mit dem anhand der Absorptionswahrscheinlichkeiten bestimmten Verhältnis $25+-31$ überein.
Dies ist möglicherweise dadurch zu erklären, dass der Inhalt des Photo-Peaks durch andere Effekte wie  Andererseits,
ist davon auszugehen, dass auch Photonen, die dem Comptoneffekt unterlagen und somit nur einen Teil ihrer Energie abgegeben haben, nochmals über den Photoeffekt
mit dem Detektor in Wechselwirkung treten, sodass der gemessene Photopeak mehr Ereignisse als erwartet enthält. Das würde auch für eine Verbreiterung des
Photopeaks sprechen, was die unerwartet große Halbwertsbreite erklären könnte. Andererseits ist auch zu nennen, dass das aufgrund der Absorptionswahrscheinlichkeiten bestimmte
Verhältnis einen so großen Fehler aufweist, das mit einer gewissen Wahrscheinlichkeit der experimentell bestimmte Wert damit übereinstimmt.\\ \\
%(d) Aktivitätsbestimmung von Barium 133
Tabelle \ref{tab:a_d_1} zeigt, dass die berechneten Aktivitäten der Probe starke Unsicherheiten aufweisen.
Der wesentliche Grund dafür sind die großen Unsicherheiten bei der Effizienzmessung, die sich in diese Ergebnisse fortpflanzen.
Ein genauer Wert für die Aktivität kann daher nicht angegeben werden.
Die Messung liefert lediglich einen Orientierungswert.
Auch mit einer präziseren Effizienzmessung würden die gesammelten Messwerte teils stark voneinander abweichen.\\ \\
% e)
Bei der Untersucung des unbekannten Strahlers auf in der Natur häufig vorkommende Zerfallsketten, konnte
ein Teil der Uran 238 Zerfallskette identifiziert werden. Allerdings konnten nicht für jedes gefundene
Nuklid alle in der Versuchsanleitung angegebenen Spektrallinien gefunden werden.(siehe Abbildung \ref{tab:nuklide})
Dabei hat sicherlich das Stören unerwünschter Bestandteile(Compton-kontinuum, Rückstreupeaks, Rauschen durch Messapparatur) bei der Suche nach den Linien eine Rolle gespielt.
Weiterhin liegen manche Linien auch außerhalb des beobachteten Bereichs, sodass diese nicht gefunden werden konnten.
Eine weiteres Problem bestand auch darin, dass für manche im Spektrum "gefundene" Linien kein genauer
Wert über den Gaussfit gefunden werden konnte da der Untergrund den Fit unmöglich machte. Diese Linien wurden
nicht in die weitere Analyse aufgenommen.
