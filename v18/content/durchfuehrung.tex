\section{Durchführung}
\label{sec:Durchführung}
Die Messungen zu dem hier beschriebenen Versuch zu Gamma-Spektroskopie mithilfe eines Germanium-Detektors,
werden mit einer bei Versuchsbeginn fertig aufgebauten Messaperatur durchgeführt. Aufgrund der angelegten Hochspannung wird
die Detektorkonstruktion permanent mithilfe von flüssigem Stickstoff gekühlt. Weiterhin ist die Apparatur nach außen hin durch
eine Bleiabschirmung vor Umgebungsstrahlung geschützt und der Detektor an sich ist nochmals durch eine Kupferabschirmung
vor Strahlung durch aüßere Bauteile der Apparatur sowie der Bleiabschirmung geschützt.\\
Die Messdaten werden mithilfe eines Computerprogramms eingelesen und liegen somit in elektronischer Form vor.\\
Es werden vier Messungen durchgeführt.\\ \\
Die erste Messung soll der Kallibrierung der Energieskala dienen, dazu wird das Spektrum eines bekannten Europium 152
Strahlers über eine Zeitspanne von $\SI{7770}{\second}$ aufgezeichnet.\\
Anschließend wird zur Untersuchung von Detektoreigenschaften das Spektrum eines unbekannten Caesium 137 Strahlers
über eine Zeitspanne von $\SI{3028}{\second}$ aufgezeichnet.\\
Danach wird das Spektrum eines Barium 133 Strahlers bei einer Zeitspanne von $\SI{3239}{\second}$ aufgezeichnet, um
Aktivität des Strahlers zu bestimmen.\\
Zuletzt wird noch das Spektrum eines unbekannten Strahlers über eine Zeitspanne von $\SI{5016}{\second}$ aufgenommen.
