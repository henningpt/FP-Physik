\section{Durchführung}
\label{sec:Durchführung}
Die Messungen zu dem hier beschriebenen Versuch zur Gamma-Spektroskopie mithilfe eines Germanium-Detektors,
werden mit einer bei Versuchsbeginn fertig aufgebauten Messapparatur durchgeführt.
Die Apparatur ist nach außen hin durch eine Bleiabschirmung von der Umgebungsstrahlung abgeschirmt und der Detektor
selbst ist nochmals durch eine Kupferabschirmung vor Strahlung durch aüßere Bauteile der Apparatur sowie der
Bleiabschirmung geschützt.\\
Die Messdaten werden mithilfe eines Computerprogramms eingelesen und liegen somit in elektronischer Form vor.\\
Es werden vier Messungen durchgeführt.\\ \\
Die erste Messung soll der Kalibrierung der Energieskala dienen, dazu wird das Spektrum eines bekannten Europium-152
Strahlers über eine Zeitspanne von $\SI{7770}{\second}$ aufgezeichnet.\\
Anschließend wird zur Untersuchung von Detektoreigenschaften das Spektrum eines unbekannten Caesium-137 Strahlers
über eine Zeitspanne von $\SI{3028}{\second}$ aufgezeichnet.\\
Danach wird das Spektrum eines Barium-133 Strahlers bei einer Zeitspanne von $\SI{3239}{\second}$ aufgezeichnet, um
Aktivität des Strahlers zu bestimmen.\\
Zuletzt wird noch das Spektrum eines unbekannten Strahlers über eine Zeitspanne von $\SI{5016}{\second}$ aufgenommen.
